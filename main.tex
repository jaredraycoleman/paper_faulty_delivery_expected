\documentclass{article}
\usepackage{graphicx} % Required for inserting images
\usepackage{authblk}
\usepackage[utf8]{inputenc}
\usepackage[english]{babel}
\usepackage[T1]{fontenc}
\usepackage{geometry}
\geometry{margin=1in}

\usepackage{amsmath}
\usepackage{algorithm}
\usepackage{algpseudocode}
\usepackage{amsthm}
\usepackage{amssymb}
\usepackage{graphicx}
\usepackage{nameref}
\usepackage{placeins}
\usepackage{hyperref}
\usepackage{biblatex}
\usepackage[dvipsnames]{xcolor}
\hypersetup{
    colorlinks,
    linkcolor={blue!80!black},
    citecolor={blue!80!black},
    urlcolor={blue!80!black}
}

\usepackage{lipsum}
\usepackage{enumitem}

% Command for writing comments
\newcommand\JC[1]{{\color{Maroon} JC: #1}}         % Jared:    Use \JC{my note}

\newcommand*{\doi}[1]{DOI: \href{http://dx.doi.org/#1}{#1}}\setitemize{noitemsep,topsep=5pt,parsep=2pt,partopsep=0pt}

% bibliography
\addbibresource{main.bib}


% SETUP
\newtheorem{theorem}{Theorem}[section]
\newtheorem{corollary}{Corollary}[theorem]
\newtheorem{lemma}[theorem]{Lemma}
\newtheorem*{problem}{Problem}

\title{Fault-Tolerant Delivery with Two Agents}
\author[1]{Jared Coleman}
\affil[1]{Loyola Marymount University \\ \texttt{jared.coleman@lmu.edu}}
\author[2]{Evangelos Kranakis}
\affil[2]{Carleton University \\ \texttt{kranakis@scs.carleton.ca}}
\author[3]{Danny Krizanc}
\affil[3]{Wesleyan University \\ \texttt{dkrizanc@wesleyan.edu}}
\author[4]{Oscar Morales-Ponce}
\affil[4]{California State University Long Beach \texttt{Oscar.MoralesPonce@csulb.edu}}

\date{\today}


\begin{document}
\maketitle

\begin{abstract}

\end{abstract}

\section{Introduction}

\section{Related Work}

\section{Model}
An agent, referred to as the \textit{starter}, begins at $(0,0)$ with an object and moves in a straight line towards $(1,0)$ to deliver it.
To handle potential failures, a second agent, called the \textit{finisher}, starts at position $(x, y)$ and is tasked with retrieving the object and completing the delivery to $(1,0)$ if necessary.

The starter may fail at discrete points $(a_0, 0), (a_1, 0), \ldots, (a_n, 0)$ with associated probabilities $p_0, p_1, \ldots, p_n$. 
A failure at $(a_i, 0)$ implies that the starter halts at that position, requiring the finisher to retrieve the object from $(a_i, 0)$. 
Both agents travel at a maximum speed of $1$ and the finisher can change direction instantaneously.

The objective is to minimize the expected delivery time of the object. We examine two communication models:
\begin{enumerate}
    \item \textbf{Face-to-Face Model:} Communication between the starter and the finisher occurs only when they occupy the same location. Consequently, if the starter fails at $(a_i, 0)$, the finisher becomes aware of this only upon arriving at $(a_i, 0)$ (or by process of elimination).
    \item \textbf{WiFi Model:} The starter and finisher can communicate at all times. In this case, the finisher is continuously informed of the starter's location and any failure points.
\end{enumerate}

\section{Face-to-Face Model}
In this section, we present results for the face-to-face model, where the finisher is only aware of the starter's location when they occupy the same point or by process of elimination (i.e., the only remaining possibility is that the starter has failed at a specific point).



\subsection{Endpoints}
We start by considering a special case where the starter fails at $(0,0)$ with probability $p$ and at $(1,0)$ with probability $q$.

\JC{Need to change below to Evangelos' model.}

Let's start by considering the case where $(x-1)^2 + y^2 \leq 1$.
In this case, the finisher can always reach the object before it reaches $(1,0)$.
Observe the point $(m, 0)$ where $m = \frac{x^2 + y^2}{2x}$ is the point where, supposing the starter does not fail, the two agents can meet at the same time (time $t = m$).
Let's consider three candidate algorithms:
\begin{enumerate}
    \item $A_0$: The finisher moves directly to $(0,0)$, then to $(1,0)$, guaranteeing a delivery time of $1+\sqrt{x^2 + y^2}$.
    \item $A_1$: The finisher moves directly to $(1,0)$. Then, with probability $p$ the delivery time is $2+\sqrt{(x-1)^2 + y^2}$, and with probability $1-p$ the delivery time is $1$.
    \item $A_m$: The finisher moves to $(m,0)$, then to $(0,0)$, then to $(1,0)$. Then with probability $p$ the delivery time is $2 + 1 + \sqrt{(x-m)^2 + y^2}$, and with probability $1-p$ the delivery time is $1$.
\end{enumerate}
Let $D_0$, $D_1$, and $D_m$ be the expected delivery times for algorithms $A_0$, $A_1$, and $A_m$, respectively.
Then $D_0 = 1 + \sqrt{x^2 + y^2}$, $D_1 = p(2 + \sqrt{(x-1)^2 + y^2}) + (1-p)$, and $D_m = p(m + 1 + \sqrt{(x-m)^2 + y^2}) + (1-p)$.

Let's consider the case where $(x-1)^2 + y^2 > 1$ (i.e., the finisher cannot reach the object before it reaches $(1,0)$).
In this case, the only two algorithms that make sense are $A_0$ and $A_1$.
The expected delivery time for $A_0$, then is better than $A_1$ if $1 + \sqrt{x^2 + y^2} < p(2 + \sqrt{(x-1)^2 + y^2}) + (1-p)$.
Solving for $y$, we find that this is true when 
\begin{equation}
    y < 2 \frac{(1-p)p(p-x)(1-p-x)}{(1-2p)^2} \label{eq:a0:a1}
\end{equation}

\begin{theorem}
    The curve defined by Equation \ref{eq:a0:a1} has an oblique asymptote at $y = \text{sgn}(x) \cdot 2 \sqrt{\frac{(1-p)p}{(1-2p)^2}}x + \frac{1}{2}$.
\end{theorem}
\begin{proof}
    \JC{This is just a sketch of the proof.}
    Let $f(x) = 2 \frac{(1-p)p(p-x)(1-p-x)}{(1-2p)^2}$.
    Taking the limit of $f'(x)$ as $x \to \infty$, yields $2 \sqrt{\frac{(1-p)p}{(1-2p)^2}}$.
    Thus, the slope of $f(x)$ approaches a constant as $x \to \infty$.
    Then we know there must exist an oblique asymptote of the form $y = mx + b$ where $m = 2 \sqrt{\frac{(1-p)p}{(1-2p)^2}}$.
    To find $b$, we must find the value of $b$ such that $\lim_{x \to \infty} f(x) - (m x + b) = 0$.
    This yields $b = \frac{1}{2}$. Thus, the oblique asymptote is $y = \text{sgn}(x) \cdot 2 \sqrt{\frac{(1-p)p}{(1-2p)^2}}x + \frac{1}{2}$.
\end{proof}


\subsection{Midpoints}
Now, we will generalize the above results to the case where the starter can fail at any point $(a, 0)$ with probability $p$ and $(b, 0)$ with probability $q$.

\JC{Add stuff from evangelos\_model.nb}


\subsection{$n$ Points}
\JC{Add stuff on dynamic programming solution (dp\_line.tex).}


\section{Wifi Model}

\subsection{Endpoints}
\JC{This case is trivial, since the finisher would know immediately if the starter did not fail at $(0,0)$. Don't need a section for it.}

\subsection{Midpoints}
\JC{Still need to do this, Mathematica doesn't want to give the answer}

\subsection{$n$ Points}
\JC{Can we come up with a dynamic programming solution?}


\section{Conclusion}


\printbibliography

\end{document}
