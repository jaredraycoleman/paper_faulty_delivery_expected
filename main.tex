\documentclass{article}
\usepackage{graphicx} % Required for inserting images
\usepackage{authblk}
\usepackage[utf8]{inputenc}
\usepackage[english]{babel}
\usepackage[T1]{fontenc}
\usepackage{geometry}
\geometry{margin=1in}

\usepackage{amsmath}
\usepackage{algorithm}
\usepackage{algpseudocode}
\usepackage{amsthm}
\usepackage{amssymb}
\usepackage{graphicx}
\usepackage{nameref}
\usepackage{placeins}
\usepackage{hyperref}
\usepackage{biblatex}
\usepackage[dvipsnames]{xcolor}
\hypersetup{
    colorlinks,
    linkcolor={blue!80!black},
    citecolor={blue!80!black},
    urlcolor={blue!80!black}
}

\usepackage{lipsum}
\usepackage{enumitem}

% Command for writing comments
\newcommand\JC[1]{{\color{Maroon} JC: #1}}         % Jared:    Use \JC{my note}

\newcommand*{\doi}[1]{DOI: \href{http://dx.doi.org/#1}{#1}}\setitemize{noitemsep,topsep=5pt,parsep=2pt,partopsep=0pt}

% bibliography
\addbibresource{main.bib}


% SETUP
\newtheorem{theorem}{Theorem}[section]
\newtheorem{corollary}{Corollary}[theorem]
\newtheorem{lemma}[theorem]{Lemma}
\newtheorem*{problem}{Problem}

\title{A Very Interesting Title}
\author[1]{Jared Coleman}
\affil[1]{Loyola Marymount University}

\date{\today}


\begin{document}
\maketitle

\begin{abstract}
\lipsum[1]
\end{abstract}


\section{Introduction}

An agent starts at $(0, 0)$ with an object and is moving in a straight line towards $(1,0)$ to deliver it.
Because the start may fail, we deploy a second agent (called the \textit{finisher}), starting at position $(x,y)$ that can pick up the object and finish the delivery to $(1,0)$.
Suppose the starter fails at $(0,0)$ with probability $p$ and at $(1,0)$ with probability $1-p$.
We only consider the object to be delivered when the finisher and the object are co-located at $(1,0)$.
Both agents always move at a maximum speed of $1$. The finisher can start, stop, and change direction instantaneously.

Let's start by considering the case where $(x-1)^2 + y^2 \leq 1$.
In this case, the finisher can always reach the object before it reaches $(1,0)$.
Observe the point $(m, 0)$ where $m = \frac{x^2 + y^2}{2x}$ is the point where, supposing the starter does not fail, the two agents can meet at the same time (time $t = m$).
Let's consider three candidate algorithms:
\begin{enumerate}
    \item $A_0$: The finisher moves directly to $(0,0)$, then to $(1,0)$, guaranteeing a delivery time of $1+\sqrt{x^2 + y^2}$.
    \item $A_1$: The finisher moves directly to $(1,0)$. Then, with probability $p$ the delivery time is $2+\sqrt{(x-1)^2 + y^2}$, and with probability $1-p$ the delivery time is $1$.
    \item $A_m$: The finisher moves to $(m,0)$, then to $(0,0)$, then to $(1,0)$. Then with probability $p$ the delivery time is $2m + 1 + \sqrt{(x-m)^2 + y^2}$, and with probability $1-p$ the delivery time is $1$.
\end{enumerate}
Let $D_0$, $D_1$, and $D_m$ be the expected delivery times for algorithms $A_0$, $A_1$, and $A_m$, respectively.
Then $D_0 = 1 + \sqrt{x^2 + y^2}$, $D_1 = p(2 + \sqrt{(x-1)^2 + y^2}) + (1-p)$, and $D_m = p(2m + \sqrt{(x-m)^2 + y^2}) + (1-p)$.


\printbibliography

\end{document}
